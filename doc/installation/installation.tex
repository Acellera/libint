%
% The Libint Installation Manual
%

\documentclass[12pt]{article}
\usepackage{html}

\begin{document}

\newcommand{\LIBINT}{{\tt LIBINT}}
\newcommand{\LIBINTv}{{\tt LIBINT} Version 1.2}

\newcommand{\libint}{{\tt libint}}
\newcommand{\libderiv}{{\tt libderiv}}
\newcommand{\librij}{{\tt libr12}}

\newcommand{\libinth}{{\tt libint.h}}
\newcommand{\libderivh}{{\tt libderiv.h}}
\newcommand{\librijh}{{\tt libr12.h}}

\newcommand{\libinta}{{\tt libint.a}}
\newcommand{\libderiva}{{\tt libderiv.a}}
\newcommand{\librija}{{\tt libr12.a}}

\newcommand{\n}{{\bf n}}
\newcommand{\A}{{\bf A}}
\newcommand{\B}{{\bf B}}
\newcommand{\C}{{\bf C}}
\newcommand{\D}{{\bf D}}

\def\bmath#1{\mbox{\boldmath$#1$}}



\begin{center}
\ \\
\vspace{2.0in}
{\bf {\Large Installation Manual for the \LIBINT\ Program Package}} \\
\vspace{0.5in}
Edward F.\ Valeev \\
\ \\
{\em Oak Ridge National Laboratory, Oak Ridge, Tennessee}\\
\vspace{0.1in}
{\em Center for Computational Molecular Science and Technology, \mbox{Georgia 
Institute of Technology,} Atlanta, Georgia 30332-0400} 
\ \\
\vspace{0.3in}
\LIBINTv \\
Created on: \today
\end{center}

\thispagestyle{empty}

\newpage
\section{Compilation Prerequisites}

The following software is needed to compile \LIBINTv:
\begin{itemize}
\item GNU C++ compiler (version 3.4 is recommended) to compile the library compiler.
\item (Optional) optimizing C++ compiler
to compile generated source. GNU C++ compiler may produce less than optimal code on some platforms.
Thus a vendor compiler is recommended (see architecture-specific notes at the end of this document).
\item Perl interpreter (version 5.005 or higher)
\item Various GNU utilies: \htmladdnormallink{{\tt
www.gnu.org}}{http://www.gnu.org}
\begin{itemize}
\item {\tt autoconf (version 2.57 or higher)}
\item {\tt make}
\item {\tt fileutils} (esp.\ {\tt install})
\end{itemize}
\item For documentation:
\begin{itemize}
\item {\tt LaTeX}
\item {\tt Doxygen}
\end{itemize}
\end{itemize}

\section{Basic Configuration and Installation}

Issue the following series of commands in the source directory to configure and build the \LIBINT\
library using default parameters and install under /usr/local:

\begin{enumerate}
\item {\tt aclocal -I lib/autoconf}
\item {\tt autoconf}
\item {\tt mkdir objdir}
\item {\tt cd objdir}
\item {\tt ../configure}
\item {\tt make}
\item {\tt make tests} (optional, but recommended)
\item {\tt make install}
\item Optional: {\tt make doc}
\end{enumerate}

\noindent
You may need to make use of one or more of the following options to
the {\tt configure} script:
\begin{itemize}
\item {\tt -}{\tt -enable-eri=N} --- Use this option to enable support for
  N-th order derivatives of electron repulsion integrals. To
  disable support for ERIs set N to 'no', or use the {\tt -}{\tt -disable-eri} option.
  By default, N=0 (i.e. no derivatives are requested).
  WARNING: No ERI derivatives can be computed yet. This functionality is forthcoming.
\item {\tt -}{\tt -enable-g12=N} --- Use this option to enable support for
  N-th order derivatives of integrals for linear G12 methods. To
  disable support for such integrals set N to 'no', or use the {\tt -}{\tt -disable-g12} option.
  By default, no linear G12 integrals are produced.
  WARNING: No derivatives can be computed yet. This functionality is forthcoming.
\item {\tt -}{\tt -with-eri-max-am=L} --- Specifies the maximum
  angular momentum level for the Gaussian basis functions
  when computing electron repulsion integrals.  By default,
  ERIs over $g$-type functions (L=4) are supported.
\item {\tt -}{\tt -with-g12-max-am=L} --- Specifies the maximum
  angular momentum level for the Gaussian basis functions
  when computing linear G12 integrals.  By default,
  integrals over $p$-type functions (L=1) are supported.
\item {\tt -}{\tt -with-float-type=TYPE} --- Specifies the C++ floating-point type
  used for integral computation. The default is to use {\tt double}.
\item {\tt -}{\tt -prefix=directory} --- Use this option if you wish to
  install the \LIBINT\ library somewhere other than the default
  directory, {\tt /usr/local}.
\item {\tt -}{\tt -with-cxx=compiler} --- Use this option to specify the location
  of the GNU C++ compiler.
\item {\tt -}{\tt -with-cxx-optflags=....} --- Specifies the optimization flags for
  the GNU C++ compiler.
\item {\tt -}{\tt -with-cxxgen=compiler} --- Use this option to specify a C++ compiler
  which will compile generated library source (see Prerequisites). By default, the GNU C++
  compiler specified with {\tt -}{\tt -with-cxx} option will be used.
\item {\tt -}{\tt -with-cxxgen-optflags=....} --- Specifies the optimization flags to be
  used for compiling the generated library source. 
\item {\tt -}{\tt -with-ranlib=ranlib} --- Use this option to specify
  a ranlib program. The default behavior is to detect an appropriate
  choice automatically.
\item {\tt -}{\tt -with-ar=archiver} --- Use this option to specify an
  archiver.  The default is to look for {\tt ar} automatically.
\item {\tt -}{\tt -with-ar-flags=options} --- Use this option to specify
  archiver command-line flags. The default is {\tt r}.
\item {\tt -}{\tt -with-perl=perl} --- Use this option to specify a
  Perl interpreter.  The default is to look for {\tt perl} automatically.
\item {\tt -}{\tt -with-debug=option} --- This option turns on debugging
  options.  If the argument is omitted, ``{\tt -g}'' will be used by default.
  To compile with optimization and debugging options, specify {\tt -}{\tt -with-debug=opt}.
\item {\tt -}{\tt -with-scratchdir=directory} --- This option specifies the location
  under which the library source will be generated and compiled. The higher the angular
  momentum the more disk space will be needed. In general, this directory should be
  on a large local disk, e.g. {\tt /tmp} or {\tt /scratch}. Default is to compile
  in a subdirectory of {\tt \$objdir}.
\end{itemize}

\section{Detailed Installation Instructions}

This section provides detailed instructions for compiling and
installing the \LIBINT\ library.  

\subsection{Step 1: Configuration}

After unpacking the code, you may need to run {\tt aclocal -I lib/autoconf; autoconf}
in order to create configure file, if you have obtained the source from CVS repository.
Next, you should create a directory to which the library will be installed.
I find it most convenient to assign a descriptive name to this diectory, e.g.
{\tt x86\_64-linux-icc90-debug}. Lastly, in the newly created directory
create a subdirectory in which compilation will occur, e.g.
{\tt x86\_64-linux-icc90-debug/obj}. The compilation directory will be referred to as {\tt \$objdir}
for the remainder of these instructions.

In {\tt \$objdir}, run the configure script found in \LIBINT\
top-level source directory.  This script will scan your system to locate
certain libraries, header files, etc. needed for complete compilation.
The script accepts a number of options, all of which are listed above.
The most important of these is the {\tt --prefix} option, which selects the
installation directory for the library, i.e. {\tt x86\_64-linux-icc90-debug} in our example.
The default {\tt -}{\tt -prefix} is {\tt /usr/local}.

$\bullet$ The configure script's {\tt -}{\tt -prefix} directory will be referred
to as {\tt \$prefix} for the remainder of these instructions.

\subsection{Step 2: Compilation}

Running {\tt make} (which must be GNU's {\tt 'make'} utility) in {\tt
\$objdir} will compile the \LIBINT\ library.

\subsection{Step 3: Testing}

There is no formal test of the generated library. There are some validation
and timing tests for ERI code under {\tt src/bin/test_eri}.

\subsection{Step 4: Installation}

Once testing is complete, installation into \$prefix is accomplished by
running {\tt make install} in {\tt \$objdir}.  The library will be
installed under {\tt \$prefix/lib} and header files under {\tt \$prefix/include}.

\subsection{Step 5: Documentation}

If your system has the appropriate utilities, you may build the package
documentation from the top-level {\tt \$objdir} by running {\tt make doc}.  
The resulting files will appear in the {\tt \$prefix/doc} area.

\subsection{Step 6: Cleaning}

All compilation-area object files can be removed to save
disk space by running {\tt make clean} in {\tt \$objdir}.

\section{Miscellaneous architecture-specific notes}
\begin{itemize}
\item None yet.
\end{itemize}


\end{document}
