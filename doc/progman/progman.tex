%
% The LIBINT Programmer's Manual
%

\documentclass[12pt]{article}
\usepackage{amsmath}
\usepackage{listings}
\lstloadlanguages{[ISO]C++}
\usepackage{courier}
\usepackage{color}
\usepackage{xcolor}

\lstset{
         basicstyle=\footnotesize\ttfamily, % Standardschrift
         %numbers=left,               % Ort der Zeilennummern
         numberstyle=\tiny,          % Stil der Zeilennummern
         %stepnumber=2,               % Abstand zwischen den Zeilennummern
         numbersep=5pt,              % Abstand der Nummern zum Text
         tabsize=2,                  % Groesse von Tabs
         extendedchars=true,         %
         breaklines=true,            % Zeilen werden Umgebrochen
         keywordstyle=\color{red},
         frame=b,         
         stringstyle=\color{white}\ttfamily, % Farbe der String
         showspaces=false,           % Leerzeichen anzeigen ?
         showtabs=false,             % Tabs anzeigen ?
         xleftmargin=17pt,
         framexleftmargin=17pt,
         framexrightmargin=5pt,
         framexbottommargin=4pt,
         %backgroundcolor=\color{lightgray},
         showstringspaces=false      % Leerzeichen in Strings anzeigen ?        
 }
\usepackage{caption}
\DeclareCaptionFont{white}{\color{white}}
\DeclareCaptionFormat{listing}{\colorbox{gray}{\parbox{\textwidth}{#1#2#3}}}
\captionsetup[lstlisting]{format=listing,labelfont=white,textfont=white}

\begin{document}

\newcommand{\LIBINT}{{\tt LIBINT}}
\newcommand{\LIBINTv}{{\tt LIBINT} Version 1.2}

\newcommand{\libint}{{\tt libint}}
\newcommand{\libderiv}{{\tt libderiv}}
\newcommand{\librij}{{\tt libr12}}

\newcommand{\libinth}{{\tt libint.h}}
\newcommand{\libderivh}{{\tt libderiv.h}}
\newcommand{\librijh}{{\tt libr12.h}}

\newcommand{\libinta}{{\tt libint.a}}
\newcommand{\libderiva}{{\tt libderiv.a}}
\newcommand{\librija}{{\tt libr12.a}}

\newcommand{\n}{{\bf n}}
\newcommand{\A}{{\bf A}}
\newcommand{\B}{{\bf B}}
\newcommand{\C}{{\bf C}}
\newcommand{\D}{{\bf D}}

\def\bmath#1{\mbox{\boldmath$#1$}}



\begin{center}
\ \\
\vspace{2.0in}
{\bf {\Large The \LIBINT\ Programmer's Manual}} \\
\vspace{0.5in}
Edward F.\ Valeev \\
\ \\
{\em Department of Chemistry, Virginia Tech, Blacksburg, Virginia 24061 USA}\\
\ \\
\vspace{0.3in}
\LIBINTv \\
Created on: \today
\end{center}

\thispagestyle{empty}

\newpage
\section{Introduction}
\LIBINT\ library contains functions to compute many-body integrals over Gaussian
functions which appear in electronic and molecular structure theories.
\LIBINTv \cite{Libint1}\ can currently compute several different types of integrals:

\begin{itemize}

\item Two-particle Coulomb (electron repulsion) integrals (ERIs). This is by
far the most common type of integrals in molecular structure theory. Two-,
three-, and four-center integrals, and their geometrical derivatives, are
supported.

\item Two-electron integrals which appear in explicitly correlated R12 (or F12) methods
with Gaussian correlation factors.\cite{Kutzelnigg85,Kutzelnigg91,Persson96} All R12 methods, such as
MP2-R12, contain terms in the wave function that depend on the interelectronic distances
$r_{ij}$ (hence the name). Appearance of several types of {\em two}-body integrals
is due to the use of the approximate resolution of the identity to reduce three- and four-body
integrals to products of simpler integrals.

\end{itemize}

A somewhat unusual feature of \LIBINT\ library is that its source code is
generated by a computer program, i.e., a {\em compiler}. The purpose of the \LIBINT\ compiler is twofold.
First, it is to eliminate the tedious process of manual writing, debugging,
and optimizing the integrals code.
Instead, a programmer provides high-level specification of
operators and recurrence relations,
and heuristics of how to apply the recurrence relations and the compiler takes care of the rest.
This currently requires some high-level programming, in a domain-specific
mini-language taylored for expressing recurrence relations. For example, the
following Obara-Saika recurrence relation,
\begin{eqnarray}
\label{eq:osvrrA}
({\bf a}+1_i\ {\bf b}| {\bf c} {\bf d})^{(m)} & = &
({\bf PA})_i ({\bf a} {\bf b}| {\bf c} {\bf d})^{(m)} +
({\bf WP})_i ({\bf a} {\bf b}| {\bf c} {\bf d})^{(m+1)} \nonumber \\
& & +
\frac{a_i}{2 \zeta} \Bigl[
({\bf a}-1_i\ {\bf b}| {\bf c} {\bf d})^{(m)} - \frac{\rho}{\zeta}
({\bf a}-1_i\ {\bf b}| {\bf c} {\bf d})^{(m+1)} \Bigr] \nonumber \\
& & +
\frac{b_i}{2 \zeta} \Bigl[
({\bf a}\ {\bf b}-1_i| {\bf c} {\bf d})^{(m)} - \frac{\rho}{\zeta}
({\bf a}\ {\bf b}-1_i| {\bf c} {\bf d})^{(m+1)} \Bigr] \nonumber \\
& & +
\frac{c_i}{2 (\zeta+\eta)}
({\bf a} {\bf b}| {\bf c}-1_i\ {\bf d})^{(m+1)} \nonumber \\
& & +
\frac{d_i}{2 (\zeta+\eta)}
({\bf a} {\bf b}| {\bf c}\ {\bf d}-1_i)^{(m+1)}
\end{eqnarray}
is specified in a high-level form in C++ as shown in Listing \ref{lst:osrrcode}.
Although the C++ code compactly represents the mathematical expression, we hope
to provide a more robust language for expressing recurrence relations in the
future.

\begin{lstlisting}[label=lst:osrrcode,caption=Example specification of an
Obara-Saika recurrence relation in \LIBINT\ compiler (see {\tt
src/bin/libint2/vrr\_11\_twoprep\_11.h}).
The corresponding mathematical expression is shown in Eq. \eqref{eq:osvrrA}]{}
  auto ABCD_m = factory.make_child(a,b,c,d,m);
  auto ABCD_mp1 = factory.make_child(a,b,c,d,m+1);
  expr_ = Vector("PA")[dir] * ABCD_m + Vector("WP")[dir] * ABCD_mp1;
  
  auto am1 = a - _1;
  if (exists(am1)) {
    auto Am1BCD_m = factory.make_child(am1,b,c,d,m);
    auto Am1BCD_mp1 = factory.make_child(am1,b,c,d,m+1);
    expr_ += Vector(a)[dir] * Scalar("oo2z") * (Am1BCD_m - Scalar("roz") * Am1BCD_mp1);
  }

  auto bm1 = b - _1;
  if (exists(bm1)) {
    auto ABm1CD_m = factory.make_child(a,bm1,c,d,m);
    auto ABm1CD_mp1 = factory.make_child(a,bm1,c,d,m+1);
    expr_ += Vector(b)[dir] * Scalar("oo2z") * (ABm1CD_m - Scalar("roz") * ABm1CD_mp1);
  }

  auto cm1 = c - _1;
  if (exists(cm1)) {
    auto ABCm1D_mp1 = factory.make_child(a,b,cm1,d,m+1);
    expr_ += Vector(c)[dir] * Scalar("oo2ze") * ABCm1D_mp1;
  }

  auto dm1 = d - _1;
  if (exists(dm1)) {
    auto ABCDm1_mp1 = factory.make_child(a,b,c,dm1,m+1);
    expr_ += Vector(d)[dir] * Scalar("oo2ze") * ABCDm1_mp1;
  }
\end{lstlisting}

Second, the goal of the compiler is to make possible tailoring the code to existing
and future computer architectures. For example, the ongoing trend in
processors is to support single instruction multiple data (SIMD)
parallelism. To effectively take advantage of the SIMD units
The \LIBINT\ compiler can easily generate vectorized code at user's request.
As the hardware evolves, manual code reengineering can be avoided --
only the compiler needs to be modified.

There are also drawbacks to the \LIBINT\ approach. First, the generated code
can be fairly large and thus take a long time to compile.
It is a relatively benign problem in practice.
Second, the \LIBINT\ compiler is a fairly complicated program, which in practice
limits the extent to which an average programmer can modify it.

\LIBINT\ currently implements recursive schemes based on Obara-Saika method\cite{Obara86} and Head-Gordon and Pople's
variation thereof.\cite{Head-Gordon88} Other recurrence relations can be easily implemented as needed.

Unlike version 1, which came as three separate interdependent libraries, version
2 of \LIBINT\ comes as a single library configured at code-generation time.
The following features of the library can be configured:
\begin{itemize}
\item support for four-center ERI, including optional support for two- and
three-center ERIs
\item derivative level for ERI
\item support for some special two-electron integrals for explicitly-correlated
integrals, e.g. of $[\hat{T}_1,\exp(- \alpha r_{12}^2)]$
\item optimization features (whether shell-sets of integrals can be unrolled,
whether to perform Common Subexpression Elimination, etc.)
\item vectorization features
\item algorithmic features (evaluation strategy)
\item API features (name prefix, FLOP counter, whether to accumulate target integrals, floating-point type,
shell ordering of Cartesian basis functions)
\item shared library support
\end{itemize}
Depending on its configuration, the library may not include all features.

\section{\label{sec:notation} Notation}

Following Obara and Saika,\cite{Obara86}
we write an {\em unnormalized primitive Cartesian} Gaussian function centered at {\bf A}\ as
\begin{eqnarray}
\phi ({\bf r}; \zeta, \n, {\bf A}) & = & (x - A_x)^{n_x} (y - A_y)^{n_y} (z - A_z)^{n_z} \nonumber \\
& & \times \exp [-\zeta({\bf r}-{\bf A})^2]\ ,
\end{eqnarray}
where {\bf r}\ is the coordinate vector of the electron, $\zeta$ is the orbital exponent, and
\n\ is a set of non-negative integers. The sum of $n_x$, $n_y$, and $n_z$ will be denoted $\lambda(\n)$
and be referred to as the angular momentum or orbital quantum number of the Gaussian function.
Hereafter \n\ will be termed the angular momentum index.
Henceforth, $n_i$ will refer to the $i$-th component of \n, where $i \in \{x, y, z\}$.
Basic vector addition rules will apply to these vector-like triads of numbers, e.g.
$\n + {\bf 1}_x \equiv \{ n_x+1, n_y, n_z\}$.

A set of $(\lambda(\n) + 1)(\lambda(\n) + 2)/2$ functions with the same $\lambda(\n)$, $\zeta$, and centered
at the common center
but with different \n\ form a {\em Cartesian shell},
or just a {\em shell}. For example, an $s$ shell ($\lambda=0$) has one function, a $p$ shell ($\lambda=1$) --
3 functions, etc.
There is no unique choice for the order of functions in shells.
The standard \LIBINT\ ordering is:
\begin{eqnarray}
p & : & p_x, p_y, p_z \nonumber \\
d & : & d_{xx}, d_{xy}, d_{xz}, d_{yy}, d_{yz}, d_{zz} \nonumber \\
f & : & f_{xxx}, f_{xxy}, f_{xxz}, f_{xyy}, f_{xyz}, f_{xzz}, f_{yyy}, f_{yyz}, f_{yzz}, f_{zzz} \nonumber \\
{\rm etc.} \nonumber
\end{eqnarray}
In general, the following loop structure can be used to generate angular momentum indices in the canonical \LIBINT\ order for all
members of a shell of angular momentum {\tt am}:
\begin{verbatim}
for(int i=0; i<=am; i++) {
  int nx = am - i;  /* exponent of x */
  for(int j=0; j<=i; j++) {
    int ny = i-j;   /* exponent of y */
    int nz = j;     /* exponent of z */
  }
}
\end{verbatim}
Other shell orderings are supported as well, e.g., those employed by the {\tt
intv3} engine in the {\tt MPQC} program, the ordering used in the {\tt GAMESS}
program, or that in the {\tt ORCA} program. These can be specified when the
library is generated (see the {\tt --with-cartgauss-ordering} configure flag).
Support of a new ordering is trivial to implement.
If your program relies on an ordering different from the above, please contact the author of \LIBINT .

The normalization constant for a primitive Gaussian $\phi ({\bf r}; \zeta, \n, {\bf A})$
\begin{eqnarray}
N(\zeta,\n) & = & \left[ \left(\frac{2}{\pi}\right)^{3/4}\frac{2^{(\lambda(\n))}\zeta^{(2\lambda(\n)+3)/4}}
                {[(2n_x-1)!!(2n_y-1)!!(2n_z-1)!!]^{1/2}} \right]
\end{eqnarray}

A contracted Gaussian function is just a linear combination of primitive Gaussians (also termed {\em primitives})
centered at the same center {\bf A} and with the same momentum indices {\bf n}
but with different exponents $\zeta_i$:
\begin{eqnarray}
\phi ({\bf r}; \bmath{\zeta}, {\bf C}, \n, {\bf A}) & = & (x - A_x)^{n_x} (y - A_y)^{n_y} (z - A_z)^{n_z} \nonumber \\
& & \times \sum_{i=1}^M C_i \exp [-\zeta_i ({\bf r}-{\bf A})^2]\ ,
\end{eqnarray}
Contracted Gaussians form shells the same way as primitives.
The contraction coefficients {\bf C} already include normalization constants so that the resulting combination
is properly normalized. Published contraction coefficients {\bf c} are linear coefficients for normalized primitives,
hence the normalization-including contraction coefficients {\bf C} have to be computed from them as
\begin{eqnarray} \label{eq:C1}
C_i & = & c_i N(\zeta_i,\n)
\end{eqnarray}
and scaled further so that the self-overlap of the contracted function is 1:
\begin{eqnarray} \label{eq:C2}
\frac{\pi^{3/2} (2n_x-1)!!(2n_y-1)!!(2n_z-1)!!}{2^{\lambda(\n)}}
\sum_{i=1}^M \sum_{j=1}^M \frac{C_i C_j }{(\zeta_i+\zeta_j)^{\lambda(\n)+3/2}} & = & 1
\end{eqnarray}

If sets of orbital exponents are used to form contracted Gaussians of one angular momentum only
then this is called a {\em segmented} contraction scheme. If there is a set of exponents that forms
contracted Gaussians of several angular momenta then such scheme is called {\em general} contraction.
Examples of basis sets that include general contractions include Atomic Natural Orbitals (ANO) sets.
\LIBINT\ was not designed to handle general contractions very well. You should use either split general contractions
into segments for each angular momentum (it's done for correlation consistent basis sets)
or use basis sets with segmented contractions only.

An integral of a two-electron operator $\hat{O}({\bf r}_1, {\bf r}_2)$ over unnormalized
primitive Cartesian Gaussians is written as
\begin{eqnarray}
\int \phi({\bf r}_1; \zeta_a, {\bf a}, {\bf A}) \phi ({\bf r}_2; \zeta_c, {\bf c}, \C) \hat{O}({\bf r}_1, {\bf r}_2)
\phi({\bf r}_1; \zeta_b, {\bf b}, \B) \phi({\bf r}_2; \zeta_d, {\bf d}, \D) d{\bf r}_1 d{\bf r}_2 \equiv ({\bf ab} |\hat{O}|{\bf cd})
\end{eqnarray}
A set of integrals $\{ ({\bf a} {\bf b}|\hat{O}({\bf r}_1, {\bf r}_2)|{\bf c} {\bf d}) \}$
over all possible combinations of functions ${\bf a} \in {\rm Shell A}$, ${\bf b} \in {\rm Shell B}$, etc.
will be termed a {\em shell-set}, or simply a {\em set}, of integrals. For example, a $(ps|sd)$ set consists of
$3 \times 1 \times 1 \times 6 = 18$ integrals.

The following definitions have been used throughout this work:
\begin{eqnarray}
\zeta & = & \zeta_a + \zeta_b \\
\eta  & = & \zeta_c + \zeta_d \\
\rho  & = & \frac{\zeta\eta}{\zeta+\eta} \\
{\bf P}& = & \frac{\zeta_a {\bf A} + \zeta_b \B}{\zeta} \\
{\bf Q}& = & \frac{\zeta_c \C + \zeta_d \D}{\eta} \\
{\bf W}& = & \frac{\zeta {\bf P} + \eta {\bf Q}}{\zeta+\eta}
\end{eqnarray}
The Boys function is defined as
\begin{eqnarray}
F_m(T) & = & \int_0^{1} dt\ t^{2m}\ \exp (-Tt^2)
\end{eqnarray}

Evaluation of integrals over functions of non-zero angular momentum starts with the
{\em auxiliary} integrals over primitive $s$-functions
defined as
\begin{eqnarray}
\label{eq:0000m}
({\bf 00}|{\bf 00})^{(m)} & = & 2 F_m(\rho |{\bf PQ}|^2) \sqrt{\frac{\rho}{\pi}}S_{12}S_{34}
\end{eqnarray}
where ${\bf PQ} = {\bf P} - {\bf Q}$ and primitive overlaps $S_{12}$ and $S_{34}$
are computed as
\begin{eqnarray}
S_{12} & = & \Bigl( \frac{\pi}{\zeta} \Bigr)^{3/2}
\exp \Bigl(-\frac{\zeta_a\zeta_b}{\zeta} |{\bf AB}|^2 \Bigr) \\
S_{34} & = & \Bigl( \frac{\pi}{\eta} \Bigr)^{3/2}
\exp \Bigl(-\frac{\zeta_c\zeta_d}{\eta} |{\bf CD}|^2 \Bigr)
\end{eqnarray}
In the evaluation of integrals over contracted functions it is convenient to
use auxiliary integrals over primitives which include contraction and normalization factors of the
target quartet $({\bf ab}|{\bf cd})$:
\begin{eqnarray} \label{eq:0000m_scaled}
({\bf 00}|{\bf 00})^{(m)} & = &  2 F_m(\rho |{\bf PQ}|^2) \sqrt{\frac{\rho}{\pi}}S_{12}S_{34}
C_1 C_2 C_3 C_4
\end{eqnarray}
where the coefficients $C_a$, $C_b$, $C_c$, and $C_d$ are
normalization-including contraction coefficients (Eqs. (\ref{eq:C1})
and (\ref{eq:C2})) for the first basis function out of each respective shell
in the target shell of integrals.

\section{Overview of \LIBINT 's API}

({\bf Note}: Depending on configuration, \LIBINT\ may also support computation of other, non-ERI types of integrals. Here I will describe
the parts of the interface which deals with ERIs only. The rest of the API is very similar to what's described here.)

\LIBINT\ library is a low-level C++ code. C linking convention is adopted to enable interoperability with other languages, such as C and FORTRAN.
For notes on how to use \LIBINT\ library from FORTRAN code, see subsection \ref{ssec:fort}.

\LIBINT\ API consists of 3 major components: type definitions, function and variable prototypes, and C preprocessor macros.
The API is described in the following header files:
\begin{itemize}
\item \libinth\ -- main header file, it includes all other headers (not generated).
\item \libintinttypesh\ -- architecture-specific definitions for long integer types (not generated).
\item \libinttypesh\ -- definitions for the integral evaluator types (generated).
\item \libintparamsh\ -- C preprocessor macros for library features (generated).
\item \libintifaceh\ -- prototypes for functions and data as well as some misc macros (generated).
\end{itemize}
\libinth is the only header that must be included explicitly in the user code.

The preprocessor macros provided by \libinth\ are necessary so that the user code can
access the library configuration parameters. For example, the library can be configured at the code generation time
to support in principle any finite angular momentum of the basis functions. The user code must still
test whether the maximum angular momentum present in the basis set exceeds the library limit.

The preprocessor value macros provided by \libinth\ are listed in Listing
\ref{lst:valmacros}
The most important macro, {\tt LIBINT2\_REALTYPE}, specifies the floating-point type
used by the library for all computations (usually, {\tt double}).
The next macro, {\tt LIBINT2\_MAX\_VECLEN} specifies the maximum vector length supported by the library. For the scalar code
this macro will be set to 1, whereas for the vectorized library this will have a value greater than
1.
The next macro, {\tt LIBINT2\_API\_PREFIX}, is used by the {\tt LIBINT2\_PREFIXED\_NAME} function macro and should not be
used explicitly (I list it here for reference).
The last three macros describe whether the library supports computation of the electron repulsion integrals,
the maximum angular momentum of the basis functions supported for the ERI code (value of 3 corresponds to support
of up to $f$ functions, etc.), and the maximum order of the
ERI derivatives which can be computed (currently only 0 is supported).

\begin{lstlisting}[label=lst:valmacros,caption=C preprocessor value macros
provided by \libinth . The angled brakets describe valid macro values.]{}
#define LIBINT2_REALTYPE <C++ floating-point type>
#define LIBINT2_MAX_VECLEN <positive integer>
#define LIBINT2_API_PREFIX <string>

/* ERI-specific macros */
#define LIBINT2_SUPPORT_ERI <0 or 1>
#define LIBINT2_MAX_AM_ERI <nonnegative integer>
#define LIBINT2_DERIV_ERI_ORDER <nonnegative integer>
\end{lstlisting}

\LIBINT\ API also specifies several functions and variables whose decralations
are listed in Listing \ref{lst:api}.
The first two functions perform static initialization and cleanup of the library.
Thus {\tt libint2\_static\_init} must be called before \LIBINT\ is used and
{\tt libint2\_static\_cleanup} must be called after \LIBINT\ is no longer needed (only one thread needs to call these functions).
The next two functions are used to initialize and deinitialize the key data
structure ({\em integral evaluator}) involved in the computation of the electron repulsion integrals.
To initialize the integral evaluator, {\tt libint2\_init\_eri} should be called with three arguments:
1) the pointer to the evaluator to be initialized; 2) the maximum angular momentum
of basis functions this object will support ({\tt max\_am} will affect the memory requirements
for the computation and therefore should be always set to the actual maximum value needed,
not the maximum value supported by the library); 3) optional pointer to the scratch buffer
which will be used to hold intermediate results. If the third argument is 0, then
the call will dynamically allocate the needed space. If the user code needs to control
memory allocation/deallocation, the scratch buffer needs to be allocated prior to the call.
{\tt libint2\_need\_memory\_eri} can be used to compute the required size of
the scratch buffer in units of {\tt LIBINT2\_REALTYPE}. Lastly, {\tt libint2\_build\_eri} is a 4-dimensional
array of pointers to functions which evaluate ERIs, e.g., {\tt libint2\_build\_eri[1][0][2][0](\&erieval)}
will compute the $(ps|ds)$ set using evaluator {\tt erieval} ({\tt erieval} must have been initialized
with {\tt libint2\_init\_eri}).

\begin{lstlisting}[label=lst:api,caption=\LIBINT\ API  functions and data.
{\tt N = LIBINT2\_MAX\_AM\_ERI + 1}.]{}
void libint2_static_init();
void libint2_static_cleanup();

/* ERI-specific API */
void libint2_init_eri(Libint_eri_t* libint, int max_am, LIBINT2_REALTYPE* buf);
void libint2_cleanup_eri(Libint_eri_t* libint);
size_t libint2_need_memory_eri(int max_am);
void (*libint2_build_eri[N][N][N][N])(Libint_eri_t *);
\end{lstlisting}

\LIBINT\ API I have described so far is very simple. However, as you may have noticed, there has been no mention of
where the basis set data is stored. This is because \LIBINT\ does not maintain the basis set information.
As part of \LIBINT 's philosophy to provide the leanest possible code, the user code
is in charge of precomputing basis set data and Boys function values and then feeding it to the evaluator object
of type {\tt Libint\_eri\_t}.

\begin{lstlisting}[label=lst:libintt,caption=``External'' parts of the
definition of the \LIBINT\ integral evaluator type.]{}
typedef struct {
  _aB_s__0__s__1___TwoERep_s__0__s__1___Ab__up_0[VECLEN];
  _aB_s__0__s__1___TwoERep_s__0__s__1___Ab__up_1[VECLEN];
  _aB_s__0__s__1___TwoERep_s__0__s__1___Ab__up_2[VECLEN];
  _aB_s__0__s__1___TwoERep_s__0__s__1___Ab__up_3[VECLEN];
  _aB_s__0__s__1___TwoERep_s__0__s__1___Ab__up_4[VECLEN];
  /* and so on until 4 * LIBINT2_MAX_AM_ERI */

  LIBINT2_REALTYPE WP_x[VECLEN], WP_y[VECLEN], WP_z[VECLEN];
  LIBINT2_REALTYPE WQ_x[VECLEN], WQ_y[VECLEN], WQ_z[VECLEN];
  LIBINT2_REALTYPE PA_x[VECLEN], PA_y[VECLEN], PA_z[VECLEN];
  LIBINT2_REALTYPE QC_x[VECLEN], QC_y[VECLEN], QC_z[VECLEN];
  LIBINT2_REALTYPE AB_x[VECLEN], AB_y[VECLEN], AB_z[VECLEN];
  LIBINT2_REALTYPE CD_x[VECLEN], CD_y[VECLEN], CD_z[VECLEN];

  LIBINT2_REALTYPE oo2z[VECLEN];
  LIBINT2_REALTYPE oo2e[VECLEN];
  LIBINT2_REALTYPE oo2ze[VECLEN];
  LIBINT2_REALTYPE roz[VECLEN];
  LIBINT2_REALTYPE roe[VECLEN];
  
  LIBINT2_REALTYPE* stack;   /* used internally */
  LIBINT2_REALTYPE* vstack;  /* used internally */
  LIBINT2_REALTYPE* targets[LIBINT2_MAX_NTARGETS_eri];
  int veclen;

  LIBINT2_UINT_LEAST64* nflops;
  int zero_out_targets;
} Libint_eri_t;
\end{lstlisting}

The definition of {\tt Libint\_eri\_t} is shown in Listing \ref{lst:libintt}.
{\tt Libint\_eri\_t} is a C structure, i.e., its members can be manipulated directly.
This is done to allow direct manipulation of {\tt Libint\_eri\_t} from
Fortran and other languages.
Note that only ERI-specific parts of the actual definition {\tt Libint\_eri\_t}
are shown in Listing \ref{lst:libintt}. Currently, the same data structure is
used for all types of evaluators (i.e. for ERI evaluators and Gaussian geminal type evaluators), i.e. {\tt Libint\_eri\_t} contains also members which are only used
for evaluation of Gaussian geminal integrals also. In the future the evaluator types will be generated
automatically and will be specific to each type of computation.

Let's look in detail at the members of {\tt Libint\_eri\_t} which must be precomputed before
calling the relevant {\tt libint2\_build\_eri} function:
\begin{itemize}
\item {\tt \_aB\_s\_\_0\_\_s\_\_1\_\_\_TwoERep\_s\_\_0\_\_s\_\_1\_\_\_Ab\_\_up\_m} --
values of auxiliary primitive integrals $({\bf 00}|{\bf 00})^{(m)}$ (Eq.
\eqref{eq:0000m}) for $0 \leq m \leq \lambda({\bf a}) + \lambda({\bf b}) +
\lambda({\bf c}) + \lambda({\bf d}) + C$, where $C = 0$ when computing ERIs, $C=1$ when computing first derivative ERIs, etc.
\item {\tt AB\_i}, {\tt CD\_i} -- cartesian components of vectors ${\bf AB} \equiv {\bf A} - {\bf B}$
and ${\bf CD} \equiv {\bf C} - {\bf D}$.
\item {\tt PA\_i}, {\tt QC\_i} -- cartesian components of vectors ${\bf PA} \equiv {\bf P} - {\bf A}$
and ${\bf QC} \equiv {\bf Q} - {\bf C}$.
\item {\tt WP\_i}, {\tt WQ\_i} -- cartesian components of vectors ${\bf WP} \equiv {\bf W} - {\bf P}$
and ${\bf WQ} \equiv {\bf W} - {\bf Q}$.
\item {\tt oo2z} -- $\frac{1}{2\zeta}$
\item {\tt oo2n} -- $\frac{1}{2\eta}$
\item {\tt oo2zn} -- $\frac{1}{2(\zeta+\eta)}$
\item {\tt roz} -- $\frac{\rho}{\zeta}$
\item {\tt ron} -- $\frac{\rho}{\eta}$
\end{itemize}
Most of these quantities are simple to evaluate. Evaluation of the Boys function needed to compute
the auxiliary integrals $({\bf 00}|{\bf 00})^{(m)}$ is more involved.
One should consult external sources for information on how
to compute it efficiently.\cite{Gill91}

After the integrals have been built, they are placed somewhere in the scratch buffer.
To recover their location, the array of pointers {\tt targets} is provided, e.g.,
the computed ERI shell-set is located at {\tt target[0]}.
ERI evaluation produces only one shell-set of integrals, but other types of computations
may produce several shell-sets of integrals at a time, e.g., usually all 12 derivative
ERI integrals are computed at the same time. That's why {\tt target} is an array of pointers, not a pointer.

Shell-sets of integrals contain integrals in ``row major'' order.\cite{KnuthACP} For example, if
the number of functions in each shell is $n_a$, $n_b$, $n_c$, and $n_d$, respectively,
then the integral $(ab|cd)$ is found at position $abcd = ( (a n_b + b) n_c + c) n_d + d$.

The rest of {\tt Libint\_eri\_t} is used to control various aspects of its behavior:
\begin{itemize}
\item {\tt veclength} is used in vectorized computation of integrals.
\item {\tt nflops} is used to count the total number of FLOPs (the library must have been configured
  with {\tt --enable-flop-counter}).
\item {\tt zero\_out\_targets} is used to zero out the target integral buffers. This is only useful
  if \LIBINT\ was configured with {\tt --enable-accum-ints}.
\end{itemize}

{\bf Note} that currently the \LIBINT\ compiler minimizes the amount of code it
generates by taking advantage of the permutational symmetry of the integrals.
This means that only certain combinations of the angular momenta can be handled.
In standard configuration \LIBINT\ can evaluate a shell quartet $({\bf ab}|{\bf cd})$ if $\lambda({\bf a}) \geq \lambda({\bf b})$, $\lambda({\bf c}) \geq \lambda({\bf d})$, and $\lambda({\bf c}) + \lambda({\bf d}) \geq \lambda({\bf a}) + \lambda({\bf b})$.
(There is also the ordering used by {\tt ORCA} program for which \LIBINT\ can be
configured). If one needs to compute a quartet that doesn't conform the rule,
e.g. of type $(pf|sd)$, permutational symmetry of integrals can be utilized to compute such quartet:
\begin{eqnarray}
(pq|rs) = (pq|sr) = (qp|rs) = (qp|sr) = (rs|pq) = (rs|qp)= (sr|pq) = (sr|qp)
\end{eqnarray}
In the case of $(pf|sd)$ shell quartet, one computes quartet $(ds|fp)$ instead, and then
permutes function indices back to obtain the desired $(pf|sd)$.

The final integrals that \LIBINT\ computes are not normalized. The best way to include the normalization
is to scale the auxiliary integrals $({\bf 00}|{\bf 00})^{(m)}$ by the normalization factors.
However, Gaussians in a shell of angular momentum $>1$ have different normalization factors.
Usually the convention is to unit-normalize only the functions which have all
quanta along one Cartesian direction, e.g., $d_{xx}, f_{xxx}$, etc. Some
programs (e.g., GAMESS) require all functions to be unit-normalized;
this can be achieved by scaling the final integrals.

\subsection{Notes on computing derivative ERIs}
Component \libderiv\ is used to evaluate derivatives of ERIs with respect to basis function positions.
Using \libderiv\ is mostly similar to how \libint\ is used. Here we only concentrate on significant
differences which have not been noted before or on aspects of use specific to \libderiv .

One quartet of ERIs $({\bf ab}|{\bf cd})$ has total of 12 first derivatives
\begin{eqnarray}
& & \frac{\partial ({\bf ab}|{\bf cd})}{\partial A_i}, \frac{\partial ({\bf ab}|{\bf cd})}{\partial B_i},
\frac{\partial ({\bf ab}|{\bf cd})}{\partial C_i},
\frac{\partial ({\bf ab}|{\bf cd})}{\partial D_i} :\quad i \in \{x,y,z\} \nonumber
\end{eqnarray}
and $12*12=144$ second derivatives, although $12*13/2=78$ derivatives are unique because of
permutation symmetry with respect to the order of taking the derivative:
\begin{eqnarray}
& & \frac{\partial^2 ({\bf ab}|{\bf cd})}{\partial A_i \partial A_j}, \frac{\partial^2 ({\bf ab}|{\bf cd})}{\partial B_i \partial B_j},
\frac{\partial^2 ({\bf ab}|{\bf cd})}{\partial C_i \partial C_j}, \frac{\partial^2 ({\bf ab}|{\bf cd})}{\partial D_i \partial D_j} :\quad
i \leq j \in \{x,y,z\} \nonumber \\
& & \frac{\partial^2 ({\bf ab}|{\bf cd})}{\partial A_i \partial B_j}, \frac{\partial^2 ({\bf ab}|{\bf cd})}{\partial A_i \partial C_j},
\frac{\partial^2 ({\bf ab}|{\bf cd})}{\partial A_i \partial D_j}, \nonumber \\
& & \frac{\partial^2 ({\bf ab}|{\bf cd})}{\partial B_i \partial C_j}, \frac{\partial^2 ({\bf ab}|{\bf cd})}{\partial B_i \partial D_j},
\frac{\partial^2 ({\bf ab}|{\bf cd})}{\partial C_i \partial D_j} : \quad i,j \in \{x,y,z\} \nonumber
\end{eqnarray}
Translational invariance of ERIs can be used to eliminate any 3 of 12 first derivatives
\begin{eqnarray} \label{eqn:TId1eri}
\frac{\partial ({\bf ab}|{\bf cd})}{\partial B_i} & = & - \frac{\partial ({\bf ab}|{\bf cd})}{\partial A_i} -
\frac{\partial ({\bf ab}|{\bf cd})}{\partial C_i} - \frac{\partial ({\bf ab}|{\bf cd})}{\partial D_i} \quad i \in \{x,y,z\}
\end{eqnarray}
and
33 of 78 second derivatives
\begin{eqnarray} \label{eqn:TId2eri_AB}
\frac{\partial^2 ({\bf ab}|{\bf cd})}{\partial A_i \partial B_j} & = & - \frac{\partial^2 ({\bf ab}|{\bf cd})}{\partial A_i \partial A_j} -
\frac{\partial^2 ({\bf ab}|{\bf cd})}{\partial A_i \partial C_j} - \frac{\partial^2 ({\bf ab}|{\bf cd})}{\partial A_i \partial D_j} \quad i,j \in \{x,y,z\} \\
\label{eqn:TId2eri_BB}
\frac{\partial^2 ({\bf ab}|{\bf cd})}{\partial B_i \partial B_j} & = & \frac{\partial^2 ({\bf ab}|{\bf cd})}{\partial A_i \partial A_j} +
\frac{\partial^2 ({\bf ab}|{\bf cd})}{\partial A_i \partial C_j} + \frac{\partial^2 ({\bf ab}|{\bf cd})}{\partial A_i \partial D_j} \nonumber \\
& & \frac{\partial^2 ({\bf ab}|{\bf cd})}{\partial A_i \partial C_j} +
\frac{\partial^2 ({\bf ab}|{\bf cd})}{\partial C_i \partial C_j} + \frac{\partial^2 ({\bf ab}|{\bf cd})}{\partial C_i \partial D_j} \nonumber \\
& & \frac{\partial^2 ({\bf ab}|{\bf cd})}{\partial A_j \partial D_i} +
\frac{\partial^2 ({\bf ab}|{\bf cd})}{\partial C_j \partial D_i} + \frac{\partial^2 ({\bf ab}|{\bf cd})}{\partial D_i \partial D_j} \quad i \leq j \in \{x,y,z\} \\
\label{eqn:TId2eri_BC}
\frac{\partial^2 ({\bf ab}|{\bf cd})}{\partial B_i \partial C_j} & = & - \frac{\partial^2 ({\bf ab}|{\bf cd})}{\partial A_i \partial C_j} -
\frac{\partial^2 ({\bf ab}|{\bf cd})}{\partial C_i \partial C_j} - \frac{\partial^2 ({\bf ab}|{\bf cd})}{\partial C_j \partial D_i} \quad i,j \in \{x,y,z\} \\
\label{eqn:TId2eri_BD}
\frac{\partial^2 ({\bf ab}|{\bf cd})}{\partial B_i \partial D_j} & = & - \frac{\partial^2 ({\bf ab}|{\bf cd})}{\partial A_i \partial D_j} -
\frac{\partial^2 ({\bf ab}|{\bf cd})}{\partial C_i \partial D_j} - \frac{\partial^2 ({\bf ab}|{\bf cd})}{\partial D_i \partial D_j} \quad i,j \in \{x,y,z\} \\
\end{eqnarray}

While \libint\ computes one target quartet at a time, \libderiv\ evaluates all
of its possible unique derivatives. There are 2 types of ``compute'' functions in \libderiv\ (see \libderivh):
\begin{verbatim}
extern void (*build_deriv1_eri[5][5][5][5])(Libderiv_t *, int);
extern void (*build_deriv12_eri[4][4][4][4])(Libderiv_t *, int);
\end{verbatim}
The former refers to functions which compute only first derivative ERIs, and the second
refers to functions which compute both first and second derivative ERIs.
The dimensions of each array are determined by the following 2 configure-time macros:
\begin{verbatim}
#define LIBDERIV_MAX_AM1 5
#define LIBDERIV_MAX_AM12 4
\end{verbatim}

Note that ``compute'' functions in \libint, {\tt build\_eri}, simply return a pointer
to the target quartet, whereas \libderiv 's functions return target data through integrals
evaluator object, {\tt Libderiv\_t}. Such objects are initialized using one of the following functions:
\begin{verbatim}
int  init_libderiv1(Libderiv_t *, int max_am, int max_num_prim_quartets,
                    int max_cart_class_size);
int  init_libderiv12(Libderiv_t *, int max_am, int max_num_prim_quartets,
                     int max_cart_class_size);
\end{verbatim}
These functions initialize objects for use with {\tt build\_deriv1\_eri} and
{\tt build\_deriv12\_eri} compute functions, respectively. It is illegal to use
a {\tt Libderiv\_t} object initialized by {\tt init\_libderiv1()} with
{\tt build\_deriv12\_eri} compute functions.
Memory requirements for initializing these two types of objects are evaluated using
\begin{verbatim}
int  libderiv1_storage_required(int max_am, int max_num_prim_quartets,
                                int max_cart_class_size);
int  libderiv12_storage_required(int max_am, int max_num_prim_quartets,
                                 int max_cart_class_size);
\end{verbatim}

Structure of {\tt Libderiv\_t} is essentially similar to {\tt Libint\_t}:
\begin{verbatim}
typedef struct {
  double *int_stack;
  prim_data *PrimQuartet;
  double *zero_stack;
  double *ABCD[12+144];
  double AB[3];
  double CD[3];
  double *deriv_classes[9][9][12];
  double *deriv2_classes[9][9][144];
  double *dvrr_classes[9][9];
  double *dvrr_stack;
  } Libderiv_t;
\end{verbatim}
User passes quartet data to \libderiv\ through {\tt PrimQuartet}, {\tt AB},
and {\tt CD}. Data is returned through member {\tt ABCD}. The dimension of {\tt ABCD}
is explicitly written as 12+144 which refer to the number of
all (including nonunique) first and second derivatives of ERIs. If a derivative index runs 
For example, {\tt ABCD[4]} and {\tt ABCD[11]} point
to derivative quartets $\frac{\partial ({\bf ab}|{\bf cd})}{\partial B_y}$ and $\frac{\partial ({\bf ab}|{\bf cd})}{\partial D_z}$, respectively.
Similarly, {\tt ABCD[13]} and {\tt ABCD[27]} refer to $\frac{\partial^2 ({\bf ab}|{\bf cd})}{\partial A_x \partial A_y}$
and $\frac{\partial^2 ({\bf ab}|{\bf cd})}{\partial A_y \partial B_x}$, respectively.

Due to the translation invariance relations and the permutational symmetry of the second derivative integrals,
some derivative quartets are not computed and thus
only some elements of this array are initialized. Eqs. (\ref{eqn:TId1eri}-\ref{eqn:TId2eri_BD})
specify how to evaluate elements which are not computed. Thus {\tt build\_deriv1\_eri()} and {\tt build\_deriv12\_eri()}
functions produce 9 and $9+45=54$ unique derivative quartets, respectively. The unique quartets and corresponding
elements of {\tt Libderiv\_t.ABCD} are listed here:
\begin{scriptsize}
\begin{eqnarray}
\frac{\partial ({\bf ab}|{\bf cd})}{\partial A_x} \quad 0 \quad \quad \frac{\partial^2 ({\bf ab}|{\bf cd})}{\partial A_y \partial A_y} \quad 25 \quad \quad 
\frac{\partial^2 ({\bf ab}|{\bf cd})}{\partial C_x \partial D_x} \quad 93 \nonumber \\
\frac{\partial ({\bf ab}|{\bf cd})}{\partial A_y} \quad 1 \quad \quad \frac{\partial^2 ({\bf ab}|{\bf cd})}{\partial A_y \partial A_z} \quad 26 \quad \quad 
\frac{\partial^2 ({\bf ab}|{\bf cd})}{\partial C_x \partial D_y} \quad 94 \nonumber \\
\frac{\partial ({\bf ab}|{\bf cd})}{\partial A_z} \quad 2 \quad \quad \frac{\partial^2 ({\bf ab}|{\bf cd})}{\partial A_y \partial C_x} \quad 30 \quad \quad 
\frac{\partial^2 ({\bf ab}|{\bf cd})}{\partial C_x \partial D_z} \quad 95 \nonumber \\
\frac{\partial ({\bf ab}|{\bf cd})}{\partial C_x} \quad 6 \quad \quad \frac{\partial^2 ({\bf ab}|{\bf cd})}{\partial A_y \partial C_y} \quad 31 \quad \quad 
\frac{\partial^2 ({\bf ab}|{\bf cd})}{\partial C_y \partial C_y} \quad 103 \nonumber \\
\frac{\partial ({\bf ab}|{\bf cd})}{\partial C_y} \quad 7 \quad \quad \frac{\partial^2 ({\bf ab}|{\bf cd})}{\partial A_y \partial C_z} \quad 32 \quad \quad 
\frac{\partial^2 ({\bf ab}|{\bf cd})}{\partial C_y \partial C_z} \quad 104 \nonumber \\
\frac{\partial ({\bf ab}|{\bf cd})}{\partial C_z} \quad 8 \quad \quad \frac{\partial^2 ({\bf ab}|{\bf cd})}{\partial A_y \partial D_x} \quad 33 \quad \quad 
\frac{\partial^2 ({\bf ab}|{\bf cd})}{\partial C_y \partial D_x} \quad 105 \nonumber \\
\frac{\partial ({\bf ab}|{\bf cd})}{\partial D_x} \quad 9 \quad \quad \frac{\partial^2 ({\bf ab}|{\bf cd})}{\partial A_y \partial D_y} \quad 34 \quad \quad 
\frac{\partial^2 ({\bf ab}|{\bf cd})}{\partial C_y \partial D_y} \quad 106 \nonumber \\
\frac{\partial ({\bf ab}|{\bf cd})}{\partial D_y} \quad 10 \quad \quad \frac{\partial^2 ({\bf ab}|{\bf cd})}{\partial A_y \partial D_z} \quad 35 \quad \quad 
\frac{\partial^2 ({\bf ab}|{\bf cd})}{\partial C_y \partial D_z} \quad 107 \nonumber \\
\frac{\partial ({\bf ab}|{\bf cd})}{\partial D_z} \quad 11 \quad \quad \frac{\partial^2 ({\bf ab}|{\bf cd})}{\partial A_z \partial A_z} \quad 38 \quad \quad 
\frac{\partial^2 ({\bf ab}|{\bf cd})}{\partial C_z \partial C_z} \quad 116 \nonumber \\
\frac{\partial^2 ({\bf ab}|{\bf cd})}{\partial A_x \partial A_x} \quad 12 \quad \quad \frac{\partial^2 ({\bf ab}|{\bf cd})}{\partial A_z \partial C_x} \quad 42 \quad \quad 
\frac{\partial^2 ({\bf ab}|{\bf cd})}{\partial C_z \partial D_x} \quad 117 \nonumber \\
\frac{\partial^2 ({\bf ab}|{\bf cd})}{\partial A_x \partial A_y} \quad 13 \quad \quad \frac{\partial^2 ({\bf ab}|{\bf cd})}{\partial A_z \partial C_y} \quad 43 \quad \quad 
\frac{\partial^2 ({\bf ab}|{\bf cd})}{\partial C_z \partial D_y} \quad 118 \nonumber \\
\frac{\partial^2 ({\bf ab}|{\bf cd})}{\partial A_x \partial A_z} \quad 14 \quad \quad \frac{\partial^2 ({\bf ab}|{\bf cd})}{\partial A_z \partial C_z} \quad 44 \quad \quad 
\frac{\partial^2 ({\bf ab}|{\bf cd})}{\partial C_z \partial D_z} \quad 119 \nonumber \\
\frac{\partial^2 ({\bf ab}|{\bf cd})}{\partial A_x \partial C_x} \quad 18 \quad \quad \frac{\partial^2 ({\bf ab}|{\bf cd})}{\partial A_z \partial D_x} \quad 45 \quad \quad 
\frac{\partial^2 ({\bf ab}|{\bf cd})}{\partial D_x \partial D_x} \quad 129 \nonumber \\
\frac{\partial^2 ({\bf ab}|{\bf cd})}{\partial A_x \partial C_y} \quad 19 \quad \quad \frac{\partial^2 ({\bf ab}|{\bf cd})}{\partial A_z \partial D_y} \quad 46 \quad \quad 
\frac{\partial^2 ({\bf ab}|{\bf cd})}{\partial D_x \partial D_y} \quad 130 \nonumber \\
\frac{\partial^2 ({\bf ab}|{\bf cd})}{\partial A_x \partial C_z} \quad 20 \quad \quad \frac{\partial^2 ({\bf ab}|{\bf cd})}{\partial A_z \partial D_z} \quad 47 \quad \quad 
\frac{\partial^2 ({\bf ab}|{\bf cd})}{\partial D_x \partial D_z} \quad 131 \nonumber \\
\frac{\partial^2 ({\bf ab}|{\bf cd})}{\partial A_x \partial D_x} \quad 21 \quad \quad \frac{\partial^2 ({\bf ab}|{\bf cd})}{\partial C_x \partial C_x} \quad 90 \quad \quad 
\frac{\partial^2 ({\bf ab}|{\bf cd})}{\partial D_y \partial D_y} \quad 142 \nonumber \\
\frac{\partial^2 ({\bf ab}|{\bf cd})}{\partial A_x \partial D_y} \quad 22 \quad \quad \frac{\partial^2 ({\bf ab}|{\bf cd})}{\partial C_x \partial C_y} \quad 91 \quad \quad 
\frac{\partial^2 ({\bf ab}|{\bf cd})}{\partial D_y \partial D_z} \quad 143 \nonumber \\
\frac{\partial^2 ({\bf ab}|{\bf cd})}{\partial A_x \partial D_z} \quad 23 \quad \quad \frac{\partial^2 ({\bf ab}|{\bf cd})}{\partial C_x \partial C_z} \quad 92 \quad \quad 
\frac{\partial^2 ({\bf ab}|{\bf cd})}{\partial D_z \partial D_z} \quad 155 \nonumber
\end{eqnarray}
\end{scriptsize}

Each derivative quartet is identical in structure to a nondifferentiated
quartet, i.e. individual integrals are arranged in a row major order. Normalization convention
for the derivative integrals is the same as for the regular ERIs.

\section{Example: using \libint}

A C++ function which uses \LIBINT\ is shown in Listing \ref{lst:usecpp}.

\begin{lstlisting}[label=lst:usecpp,caption=Using \LIBINT\ from C++.]{}
#include <iostream>
#include <algorithm>
#include <libint2.h>

using namespace std;

/** This function evaluates ERI over 4 primitive Gaussians.
    See doc/sample/test_eri.cc for an example of how to deal with
    contracted Gaussians.
    
    For simplicity, many details are omitted here, e.g. normalization.
  */
    void
    compute_eri(unsigned int am1, double alpha1, double A[3],
                unsigned int am2, double alpha2, double B[3],
                unsigned int am3, double alpha3, double C[3],
                unsigned int am4, double alpha4, double D[3]
               )
{
    // I will assume that libint2_static_init() has been called elsewhere!

    // The ERI evaluator would normally be allocated once per calculation,
    // not once every integral shell-set
    Libint_eri_t erieval;
    const unsigned int max_am = max(max(am1,am2),max(am3,am4));
    libint2_init_eri(&erieval,max_am,0);
    // if have support for contracted integrals, set the contraction length to 1
#if LIBINT_CONTRACTED_INTS
    erieval.contrdepth = 1;
#endif

    //
    // Compute requisite data -- many of these quantities would be precomputed
    // for all nonnegligible shell pairs somewhere else
    //
    const double gammap = alpha1 + alpha2;
    const double Px = (alpha1*A[0] + alpha2*B[0])/gammap;
    const double Py = (alpha1*A[1] + alpha2*B[1])/gammap;
    const double Pz = (alpha1*A[2] + alpha2*B[2])/gammap;
    const double PAx = Px - A[0];
    const double PAy = Py - A[1];
    const double PAz = Pz - A[2];
    const double PBx = Px - B[0];
    const double PBy = Py - B[1];
    const double PBz = Pz - B[2];
    const double AB2 = (A[0]-B[0])*(A[0]-B[0])
                     + (A[1]-B[1])*(A[1]-B[1])
                     + (A[2]-B[2])*(A[2]-B[2]);
    
    erieval.PA_x[0] = PAx;
    erieval.PA_y[0] = PAy;
    erieval.PA_z[0] = PAz;
    erieval.AB_x[0] = A[0] - B[0];
    erieval.AB_y[0] = A[1] - B[1];
    erieval.AB_z[0] = A[2] - B[2];
    erieval.oo2z[0] = 0.5/gammap;
    
    const double gammaq = alpha3 + alpha4;
    const double gammapq = gammap*gammaq/(gammap+gammaq);
    const double Qx = (alpha3*C[0] + alpha4*D[0])/gammaq;
    const double Qy = (alpha3*C[1] + alpha4*D[1])/gammaq;
    const double Qz = (alpha3*C[2] + alpha4*D[2])/gammaq;
    const double QCx = Qx - C[0];
    const double QCy = Qy - C[1];
    const double QCz = Qz - C[2];
    const double QDx = Qx - D[0];
    const double QDy = Qy - D[1];
    const double QDz = Qz - D[2];
    const double CD2 = (C[0]-D[0])*(C[0]-D[0])
                     + (C[1]-D[1])*(C[1]-D[1])
                     + (C[2]-D[2])*(C[2]-D[2]);
    
    erieval.QC_x[0] = QCx;
    erieval.QC_y[0] = QCy;
    erieval.QC_z[0] = QCz;
    erieval.CD_x[0] = C[0] - D[0];
    erieval.CD_y[0] = C[1] - D[1];
    erieval.CD_z[0] = C[2] - D[2];
    erieval.oo2e[0] = 0.5/gammaq;
    
    const double PQx = Px - Qx;
    const double PQy = Py - Qy;
    const double PQz = Pz - Qz;
    const double PQ2 = PQx*PQx + PQy*PQy + PQz*PQz;
    const double Wx = (gammap*Px + gammaq*Qx)/(gammap+gammaq);
    const double Wy = (gammap*Py + gammaq*Qy)/(gammap+gammaq);
    const double Wz = (gammap*Pz + gammaq*Qz)/(gammap+gammaq);
    
    erieval.WP_x[0] = Wx - Px;
    erieval.WP_y[0] = Wy - Py;
    erieval.WP_z[0] = Wz - Pz;
    erieval.WQ_x[0] = Wx - Qx;
    erieval.WQ_y[0] = Wy - Qy;
    erieval.WQ_z[0] = Wz - Qz;
    erieval.oo2ze[0] = 0.5/(gammap+gammaq);
    erieval.roz[0] = gammapq/gammap;
    erieval.roe[0] = gammapq/gammaq;
    
    double K1 = exp(-alpha1*alpha2*AB2/gammap);
    double K2 = exp(-alpha3*alpha4*CD2/gammaq);
    double pfac = 2*pow(M_PI,2.5)*K1*K2/(gammap*gammaq*sqrt(gammap+gammaq));

    //
    // calc_f (not shown here) evaluates Boys function F_m for all m in [0,am]
    //
    unsigned int am = am1 + am2 + am3 + am4;
    double* F = init_array(am+1);
    calc_f(F,am,PQ2*gammapq);

    // (00|00)^m = pfac * F_m
    erieval.LIBINT_T_SS_EREP_SS(0)[0] = pfac*F[0];
    erieval.LIBINT_T_SS_EREP_SS(1)[0] = pfac*F[1];
    erieval.LIBINT_T_SS_EREP_SS(2)[0] = pfac*F[2];
    erieval.LIBINT_T_SS_EREP_SS(3)[0] = pfac*F[3];
    erieval.LIBINT_T_SS_EREP_SS(4)[0] = pfac*F[4];
    // etc.

    // compute ERIs
    libint2_build_eri[am1][am2][am3][am4](&erieval);

    // Print out the integrals
    const double* eri_shell_set = erieval.targets[0];
    const unsigned int n1 = (am1 + 1)(am1 + 2)/2;
    const unsigned int n2 = (am2 + 1)(am2 + 2)/2;
    const unsigned int n3 = (am3 + 1)(am3 + 2)/2;
    const unsigned int n4 = (am4 + 1)(am4 + 2)/2;
    for(int a=0; a<n1; a++) {
      for(int b=0; b<n2; b++) {
        for(int c=0; c<n3; c++) {
          for(int d=0; d<n4; d++) {
            cout << "a = " << a
                 << "b = " << b
                 << "c = " << c
                 << "d = " << d
                 << "(ab|cd) = " << *eri_shell_set;
            ++eri_shell_set;
          }
        }
      }
    }

    libint2_cleanup_eri(&erieval);
}
\end{lstlisting}

To see how to use \LIBINT\ for efficient computation of integrals over
contracted basis function refer to the sample code found in
{\tt tests/eri} directory.

\section{\label{sec:fort} Notes on using \LIBINT\ from Fortran}

Although \LIBINT\ source is written in C++, it should be possible to use the library
from Fortran programs. (Un)fortunately, I am not a Fortran programmer. Thus I can
only provide general guidelines here.

One of the main issues is the number of Fortran standards available. The most recent standard, Fortran 2003,
seems to have the best support for interoperability with C programs, but its compiler support is stil lacking.
Unfortunately, the most popular standard, Fortran 77, is also the most restrictive.

In general, C functions can be easily called from Fortran programs, but sharing data structures
is not straightforward. Thus the main culprit is how to modify  {\tt Libint\_eri\_t} objects
from Fortran. Fortran 2003 provides direct support for binding C data structures to
Fortran types. Older Fortran standards can access C data structures indirectly, via
common blocks. An (non-working) example of how a Fortran subroutine can manipulate {\tt Libint\_eri\_t}
is shown in Listing \ref{lst:usefort}. The
common block {\tt erieval} referened in that Listing is created by declaring
in C++ a global variable as follows:
\begin{verbatim}
extern Libint_eri_t erieval;
\end{verbatim}

\begin{lstlisting}[label=lst:usefort,caption=Accesing {\tt Libint\_eri\_t}
structure from a Fortran code.]{}
       c assuming that LIBINT2_REALTYPE is 8-bytes long
       real(8) F0(1)
       real(8) F1(1)
       real(8) F2(1)
       real(8) F3(1)
       real(8) F4(1)
       etc.
       real(8) WP_x(1), WP_y(1), WP_z(1)
       real(8) WQ_x(1), WQ_y(1), WQ_z(1)
       real(8) PA_x(1), PA_y(1), PA_z(1)
       real(8) QC_x(1), QC_y(1), QC_z(1)
       real(8) AB_x(1), AB_y(1), AB_z(1)
       real(8) CD_x(1), CD_y(1), CD_z(1)
       real(8) oo2z(1), oo2e(1), oo2ze(1), roz(1), roe(1)
       c in 64-bit environment pointers are 8-bytes long
       c assuming LIBINT2_MAX_NTARGETS is 10
       integer(8) targets(10)
       integer(4) veclength
       integer(8) nflops
       integer(4) zero_out_targets
       c common erieval represents the C object erieval
       common/erieval/ F0, F1, ... , WP_x, WP_y, etc.

       c now can access elements of erieval
       AB_x[0] = Ax[0] - Bx[0]
       etc.
\end{lstlisting}

The caveat of accessing C data structures from Fortran programs is that the members of the data structure must be
declared in the common block in the exact order in which they appear in the definition of {\tt Libint\_eri\_t}.
The actual definition of {\tt Libint\_eri\_t} in \libinttypesh\ must always be consulted.

Calling \LIBINT\ functions from Fortran should be straightforward. Using the function pointer array {\tt libint2\_build\_eri} is probably
not feasible in older Fortran standards, but perhaps can be accomplished in Fortran 2003. Actual function names must be used instead, i.e.,
a C++ expression
\begin{verbatim}
libint2_build_eri[1][0][2][0](&erival);
\end{verbatim}
will be replaced with a Fortran expression (I'm not sure how to pass pointer to erieval to the function from Fortran!)
\begin{verbatim}
_aB_p__0__d__1___TwoERep_s__0__s__1___Ab__up_0()
\end{verbatim}
where {\tt \_aB\_p\_\_0\_\_d\_\_1\_\_\_TwoERep\_s\_\_0\_\_s\_\_1\_\_\_Ab\_\_up\_0} is the name of the function
to which {\tt libint2\_build\_eri[1][0][2][0]} points.

Please send in your comments on how to actually make \LIBINT\ work from Fortran.

\bibliographystyle{unsrt}
\bibliography{refs}

\end{document}
